Avec les récents évènements, les français s'inquiètent de plus en plus de la dépréciation probable de leur niveau de vie face à l'inflation des dernières années. Afin d'aider les français et surtout le gouvernement à suivre ce niveau de vie, les institutions compétentes, notamment l'Institut National des Statistiques et Études Économiques (INSEE), ont effectué plusieurs études à l'échelle nationale concernant divers indicateur de la santé économique du pays. Ces études complètes permettent ainsi d'apprécier le salaire moyen équivalent temps plein, le niveau de vie médian, la répartition des salaires, et bien d'autres indicateurs, à différentes échelles. 

Cependant, il apparaît à la lecture de ces études qu'il manque quelques éléments pour prendre en compte les réalités économiques des territoires.

