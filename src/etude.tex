\documentclass[a4paper, french, 11pt]{elsarticle}

\usepackage[utf8]{inputenc}
\usepackage[T1]{fontenc}
\usepackage[french]{babel}
\usepackage{fullpage}
\usepackage{amsmath}
\usepackage{amssymb}
\usepackage{eurosym}
\usepackage{booktabs}
\usepackage{hyperref}
\usepackage[noabbrev]{cleveref}

\newdefinition{definition}{Définition}

\begin{document}

\begin{frontmatter}

    \title{Une étude de l'impact du loyer sur le reste à vivre français à l'échele départementale}
\author[1]{Quentin Haenn\corref{cor1}}
\ead{quentin.haenn@ensma.fr}
\ead[url]{https://github.com/quentinhaenn}

\cortext[cor1]{Joindre les auteurs}

\affiliation[1]{organization={LIAS Laboratory},
    addressline={1 Av. Clément Ader},
    postcode={86360},
    city={Chasseneuil-du-Poitou},
    country={France}
}

\begin{abstract}
    À travers cette étude j'essaie de déterminer à quel point la situation géographique peut être un frein à la vie des ménages au travers de l'impact du loyer sur le reste à vivre de ces derniers. Cette étude permet ainsi d'étendre les études menées sur le pouvoir d'achat, en raffinant à l'échelle départementale afin de gommer les effets des extrêmes.
\end{abstract}

\begin{keyword}
    statistiques locales \sep reste à vivre \sep France \sep département \sep inégalités \sep logement
\end{keyword}

\end{frontmatter}

\section{Introduction}

Avec les récents évènements, les français s'inquiètent de plus en plus de la dépréciation probable de leur niveau de vie face à l'inflation des dernières années. Afin d'aider les français et surtout le gouvernement à suivre ce niveau de vie, les institutions compétentes, notamment l'Institut National des Statistiques et Études Économiques (INSEE), ont effectué plusieurs études à l'échelle nationale concernant divers indicateur de la santé économique du pays. Ces études complètes permettent ainsi d'apprécier le salaire moyen équivalent temps plein, le niveau de vie médian, la répartition des salaires, et bien d'autres indicateurs, à différentes échelles. 

Cependant, il apparaît à la lecture de ces études qu'il manque quelques éléments pour prendre en compte les réalités économiques des territoires.



\section{Définitions}

Afin de bien cerner les différents indicateurs mis en avant par les précédentes études et leur impact sur la notre il convient de les définir.

\subsection{Définitions sémantiques}

\begin{definition}[Salaire équivalent temps plein (EQTP)]\label{def:salaire-EQTP}
    Le salaire en équivalent temps plein (EQTP) est un salaire converti à un temps plein pendant toute l'année, quel que soit le volume de travail effectif. Par exemple, pour un agent ayant occupé un poste de travail pendant six mois à 80 \% et ayant perçu un total de 10 000 euros, le salaire en EQTP est de 10 000/(0,5*0,8) = 25 000 euros par an. 

Pour calculer le salaire moyen en EQTP ou sa distribution, tous les postes y compris les postes à temps partiel sont pris en compte au prorata de leur volume de travail effectif (soit 0,5*0,8=0,4 EQTP dans l'exemple précédent). 
\end{definition}

\begin{definition}[Revenu disponible]\label{def:revenu-disponible}
Le revenu disponible est le revenu à la disposition du ménage pour consommer et épargner. Il comprend les revenus d'activité nets des cotisations sociales, les indemnités de chômage, les retraites et pensions, les revenus du patrimoine (fonciers et financiers) et les autres prestations sociales perçues, nets des impôts directs.

Ces derniers incluent l'impôt sur le revenu, la taxe d’habitation, la contribution sociale généralisée – CSG –, contribution à la réduction de la dette sociale – CRDS – et les prélèvements sociaux sur les revenus du patrimoine. Il comprend une partie du solde des transferts inter ménages. 
\end{definition}

\begin{definition}[Niveau de vie]\label{def:niveau-vie}
    Le niveau de vie est égal au revenu disponible du ménage divisé par le nombre d'unités de consommation (UC). Le niveau de vie est donc le même pour tous les individus d'un même ménage. Le niveau de vie correspond à ce qu'Eurostat nomme \og revenu disponible équivalent \fg. 

Les unités de consommation sont généralement calculées selon l'échelle d'équivalence dite de l'OCDE modifiée qui attribue 1 UC au premier adulte du ménage, 0,5 UC aux autres personnes de 14 ans ou plus et 0,3 UC aux enfants de moins de 14 ans.
\end{definition}

\begin{definition}[Loyer équivalent par commune]\label{def:loyer-eq-commune}
Loyer moyen par commune déterminé par une étude de l'Agence Natione pour l'information sur le Logement (ANIL).
\end{definition}

\begin{definition}[Loyer moyen par département]\label{def:loyer-moyen-departement}
    Loyer obtenu en moyennant tous les loyers des communes d'un même département.
\end{definition}

Bien évidemment cette dernière définition comporte un certain nombre de problèmes, notamment qu'elle demeure très sensible aux extrêmes. Cette sensibilité se ressent particulièrement dans les départements où une grande agglomération concentre les prix les plus élevés. Afin de quantifier cette dépendance aux extrêmes un nouvel indicateur sera défini. 

\subsection{Définitions mathématiques}

\begin{definition}[Reste à vivre moyen à partir du SMEQTP]
    En partant des \cref{def:salaire-EQTP,def:revenu-disponible}, il est possible de calculer un reste à vivre. On note $\overline{R_{EQTP}}$ le reste à vivre moyen à partir du salaire moyen EQTP $\overline{S_{EQTP}}$ et du loyer moyen mensuel retenu pour le calcul $\overline{L}$
    \begin{equation}
        \overline{\mathrm{R}_{EQTP}} = \overline{\mathrm{S}_{EQTP}} - \overline{L}
    \end{equation}
\end{definition}

\begin{definition}[Reste à vivre à partir du niveau de vie médian]
    En se basant cette fois sur les \cref{def:niveau-vie,def:revenu-disponible}, nous pouvons calculer un second reste à vivre, plus précis cette fois. En reprenant les notations précédentes et en introduisant $\hat{N}$ le niveau de vie médian, nous obtenons :
    \begin{equation}
        \hat{R_{N}} = \hat{N} - \overline{L}
    \end{equation}
\end{definition}

\begin{definition}[Reste à vivre à partir d'un salaire fixe]
    À partir du même salaire net il est possible d'exacerber les inégalités territoriales. Partant de ce principe on calcule donc un dernier reste à vivre à partir d'un salaire fixe de 2400 \euro \,net de charges.
    \begin{equation}
        R_{2400} = 2400 - \overline{L} - I(2400)
    \end{equation}
    Où $I(x)$ est la valeur des impôts sur le revenu prélevé à la source pour un revenu $x$.
\end{definition}

\section{Sources}

La description complète des sources de données est fournie dans cette sections. Les sources principales sont l'INSEE et l'ANIL qui nous permettent de récupérer 4 fichiers décrivant les salaires, niveaux de vie médian et loyers d'appartement et maison. 

\subsection{Données INSEE}

Les données provenant de l'INSEE décrivent le salaire moyen et le niveau de vie médian par département. Un exemple des données récupérées est donné par les \cref{table:niv-vie,table:salaire-EQTP}. Ces données sont très simples et ne comportent que 3 colonnes, qui s'expliquent naturellement au regard des définitions précédentes.

\begin{table}[h]
    \centering
    \begin{tabular}{llr}
\toprule
\textbf{Code département} & \textbf{Département} &  \textbf{Niveau de vie annuel médian} \\
\midrule
01               &         Ain &                        24030 \\
02               &       Aisne &                        20300 \\
03               &      Allier &                        20990 \\
\bottomrule
\end{tabular}

    \caption{Extrait des données INSEE concernant le niveau de vie médian par département\label{table:niv-vie}}
\end{table}


\begin{table}[h]
    \centering
    \begin{tabular}{llr}
\toprule
\textbf{Code département} & \textbf{Département} &  \textbf{Salaire net moyen mensuel en EQTP} \\
\midrule
01               &         Ain &                               2210 \\
02               &       Aisne &                               2040 \\
03               &      Allier &                               1990 \\
\bottomrule
\end{tabular}

    \caption{Extrait des données INSEE concernant le salaire moyen EQTP par département\label{table:salaire-EQTP}}
\end{table}

Nous pouvons maintenant étudier les données de l'ANIL, qui sont autrement plus complexes.

\subsection{Données ANIL}

Les données ANIL sont à disposition sur le site OpenData du gouvernement français. Nous récupérons ici les résultats de leur étude sur les loyers moyens par commune (INSERER LIEN). La notice explicative de leur méthodologie peux se trouver ICI pour plus d'explications. Il convient de noter également les limites qu'ils ont eux même inclus dans leur étude et leur notice. 

Les données sont établies pour trois types de biens immobiliers distincts :
\begin{enumerate}
    \item Les appartements de 1-2 pièces, avec une superficie typique de 37m$^2$
    \item Les appartements de 3 pièces et plus, avec une superficie typique de 72m$^2$
    \item Les maisons, avec une superficie typique de 92m$^2$
\end{enumerate}

Il est à noter que les appartements sont également regroupés sous une même appellation qui regroupe tous les types d'appartement, avec une surface typique de 52m$^2$.

Tous les indicateurs de loyers ont été établis pour des biens non meublés, et sont fournis charges comprises.

Un extrait des données est fourni par les 

\section{Résultats}

\input{parts/resultats.tex}

\section{Limites}

\input{parts/limites.tex}

\section{Conclusion}

\input{parts/conclusion.tex}


\end{document}